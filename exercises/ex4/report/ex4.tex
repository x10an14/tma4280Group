%Exercise 4 LaTeX report for TMA 4280
\documentclass[fontsize=11pt,paper=a4,titlepage]{report}
% \usepackage{float} %dunno yet??, probably replaced by amsfonts
\usepackage{listings}
\usepackage[usenames,dvipsnames]{color}		%For the SkyBlue background color for lstlistings
\usepackage{mathtools}
\usepackage{amsfonts,amsmath,amssymb,amsthm}	%For \mathbb
% \usepackage{caption}	%Dunno yet
\usepackage{todonotes}	%For \todo
\usepackage{tabularx}	%for tablecontents wrapping inside cell, instead of cell breaking page width.
\usepackage{verbatim}
\usepackage[margin=3cm]{geometry}

\newcommand*\Laplace{\mathop{}\!\mathbin\bigtriangleup}

\lstset{ %
language=C,							% choose the language of the code
basicstyle=\footnotesize,			% the size of the fonts that are used for the code
numbers=left,						% where to put the line-numbers
numberstyle=\footnotesize,			% the size of the fonts that are used for the line-numbers
stepnumber=1,						% the step between two line-numbers. If it is 1 each line will be numbered
numbersep=5pt,						% how far the line-numbers are from the code
backgroundcolor=\color{SkyBlue},	% choose the background color. You must add \usepackage{color}
showspaces=false,					% show spaces adding particular underscores
showstringspaces=false,				% underline spaces within strings
showtabs=false,						% show tabs within strings adding particular underscores
frame=single,						% adds a frame around the code
tabsize=4,							% sets default tabsize to 4 spaces
captionpos=b,						% sets the caption-position to bottom
breaklines=true,					% sets automatic line breaking
breakatwhitespace=false,			% sets if automatic breaks should only happen at whitespace
escapeinside={\%*}{*)}				% if you want to add a comment within your code
}

\lstset{literate=
	{á}{{\'a}}1 {é}{{\'e}}1 {í}{{\'i}}1 {ó}{{\'o}}1 {ú}{{\'u}}1
	{Á}{{\'A}}1 {É}{{\'E}}1 {Í}{{\'I}}1 {Ó}{{\'O}}1 {Ú}{{\'U}}1
	{à}{{\`a}}1 {è}{{\'e}}1 {ì}{{\`i}}1 {ò}{{\`o}}1 {ù}{{\`u}}1
	{À}{{\`A}}1 {È}{{\'E}}1 {Ì}{{\`I}}1 {Ò}{{\`O}}1 {Ù}{{\`U}}1
	{ä}{{\"a}}1 {ë}{{\"e}}1 {ï}{{\"i}}1 {ö}{{\"o}}1 {ü}{{\"u}}1
	{Ä}{{\"A}}1 {Ë}{{\"E}}1 {Ï}{{\"I}}1 {Ö}{{\"O}}1 {Ü}{{\"U}}1
	{â}{{\^a}}1 {ê}{{\^e}}1 {î}{{\^i}}1 {ô}{{\^o}}1 {û}{{\^u}}1
	{Â}{{\^A}}1 {Ê}{{\^E}}1 {Î}{{\^I}}1 {Ô}{{\^O}}1 {Û}{{\^U}}1
	{œ}{{\oe}}1 {Œ}{{\OE}}1 {æ}{{\ae}}1 {Æ}{{\AE}}1 {ß}{{\ss}}1
	{ç}{{\c c}}1 {Ç}{{\c C}}1 {ø}{{\o}}1 {å}{{\r a}}1 {Å}{{\r A}}1
	{€}{{\EUR}}1 {£}{{\pounds}}1
}
 %config.tex file in same directory

\begin{document}

\begin{center}

%\lstlistoflistings
% \listoffigures
% \listoftables

{\huge Problem set 4}\\[0.5cm]

\textsc{\LARGE TMA4280}\\[0.5cm]
\textsc{\large Introduction to supercomputing -}\\
\textsc{\large Mandatory problem set}\\[0.6cm]

\begin{table}[h]
\centering
\begin{tabular}{ccc}
	\textsc{Christian CHAVEZ}	&	\textsc{Mireia DUASO}	&	\textsc{Erlend SIGHOLT}
\end{tabular}
\end{table}

\large{\today}
\vfill
\section*{Abstract}
\end{center}

This report describes a program made to compute the sum of a Hyperharmonic series, through a finite approximation, and comparing said sum to the known convergence of the infinite series.

It covers the serial implementation of an algorithm, and exploitation of parallelization through the OpenMP API and OpenMPI Library, in order to improve the efficiency and speed of the program.

Finally it covers the results from utilizing parallelization, and analyzes the utility of using parallelization to speed up this task.

\addtocounter{chapter}{1}

\clearpage
\section{Introduction}

For the programming belonging to this problem set, see $\textit{ex4.c}$ file in
the zipped archive attached to this report. This program calculates the
difference between the sum of a finite Hyperharmonic series and the convergence
sum for an infinite Hyperharmonic series. This program is able to compute the
difference between the two in the following scenarioes listed below in
table~\ref{tab:RunModes}.

\begin{table}[h]
	\begin{tabularx}{\linewidth}{c|X|X|}
			OpenMP/MPI	& OpenMP off & OpenMP on	\\ \hline
			MPI off		& Using one thread and a single process for running the
computations serially on one processing core. & Using one process with multiple
threads on a single processing core for running the computations concurrently
across the threads. \\ \hline
			MPI on		& Using multiple processes with a single thread each,
potentially running on multiple cores concurrently for the programs
computations. & Using multiple threads in multiple processes running on
potentially multiple cores concurrently for the programs computations. \\
\hline
	\end{tabularx}
	\caption{Different runmodes for the program in this problem set.}
	\label{tab:RunModes}
\end{table}

\section{Series $S_n$}

Below, in equation~\ref{eq:HyperSeries}, the form of a~\textit{Hyperharmonic}
series is shown.

\begin{equation}
	\sum_{i=1}^{\infty} \frac{1}{i^p}
	\label{eq:HyperSeries}
\end{equation}

However, this problem set only asks for the sum of a certain subset of
Hyperharmonic series, when $p = 2$. Below, equation~\ref{eq:HyperSum} shows the 
$n$th partial sum expression of Hyperharmonic series.

\begin{equation}
	S_n = \sum_{i=1}^{n} \frac{1}{i^2}
	\label{eq:HyperSum}
\end{equation}

This problem set asks for a program that can calculate the sum of Hyperharmonic
series of the form shown in equation~\ref{eq:HyperSeries}, of finite length $n$.

Since the problem set wants us to develop a program which can concurrently
calculate this sum in parallel on $P = 2^m$ discrete processors, this problem
set also specifies that $n$ should follow certain conditions; $n = 2^k$, where
$k$ is given as parameter on the form $\{k \in \mathbb{N} : k \in [3, 14]\}$.

\begin{equation}
	\frac{\pi^2}{6} = \lim_{n \to \infty}S_n = \sum_{i=1}^{\infty} \frac{1}{i^2}
	\label{eq:HyperActualSum}
\end{equation}

So, after having detailed that $S = \lim_{n \to \infty}S_n$ in the above
equation~\ref{eq:HyperActualSum}, we see that its sum converges to $\frac{\pi^2}{6}$. 
It is the difference between this sum and the $n$th partial sum of Hyperharmonic 
series the problem set is interested in, and our program calculates both serially 
and in parallel.

\section{Concurrency Implementation}

The program was developed in iterations. The first iteration ran on a single
thread in a single process on a single processor core, in a non-concurrent
sequential fashion. The second iteration introduced running the program as
mentioned in the top-right cell of table~\ref{tab:RunModes}. This was enabled by
putting the following line into our \textit{ex4.c} file.

\lstinputlisting[firstline=18,lastline=18,firstnumber=18,label=lst:OpenMP]{
../ex4.c}

The third iteration focused on enabling the program to run in the runmode
corresponding to the bottom-right cell of table~\ref{tab:RunModes}.
Since OpenMP is enabled through the use of pragmas, making the code run with MPI
was then only a matter of choosing the right compiler and compilation switches.
Hence, in effect the third iteration only focused on implementing MPI into the
program.

To utilize MPI, the program does need to use a subset of the available MPI
function calls; $\textit{MPI\_Scatter()}$, $\textit{MPI\_Send()}$,
$\textit{MPI\_Reduce()}$, $\textit{MPI\_Gather()}$, $\textit{MPI\_Init()}$,
$\textit{MPI\_Finalize()}$, and so on. Not all of these are necessary, but a
minimum subset of these are. Such as, $\textit{MPI\_Init()}$ and $\textit{
MPI\_Finalize()}$, and then one of the $\textit{MPI\_Scatter()}$ or
$\textit{MPI\_Send()}$ alternatives to send data among the processes used by
MPI. Likewise there are multiple alternatives for $\textit{MPI\_Gather()}$ and
$\textit{MPI\_Reduce()}$, but you only need one of the alternatives to receive
data from the other processes used by MPI.

In our solution, implemented in the attached code, we utilized the following MPI
functions:

\begin{itemize}
	\item{$\textit{MPI\_Init()}$ and $\textit{
MPI\_Finalize()}$ through the provided \textit{common.c} \textit{common.h}
framework.}
	\item{$\textit{MPI\_Scatter()}$ and $\textit{MPI\_Reduce()}$ were used to
distribute 	and gather data respectively, and $\textit{MPI\_Reduce()}$ also
aided us by performing the summing necessary between the processes for us,
storing the result in a specified variable.}
\end{itemize}

These function calls are considered more convenient considering the purpose of
this problem set. This due to the fact that the problem is relatively simple,
and easy to load-balance. Hence, no complicated send/receive inter-communication
between the processes is needed to coordinate the concurrency. All the summing
is done independently of any other summing (besides the final summing performed
by $\textit{MPI\_Reduce()}$).

MPI could also have been used through the use of the $\textit{MPI\_Send()}$
and/or $\textit{MPI\_Gather()}$ functions. This would have provided more
detailed control over the data (location and communication), but this is not
necessary for solving this problem and might actually be harmful to efficiency.
This is due to potential added overhead for communication and variable
assignment. (It is here assumed that the implementation of
$\textit{MPI\_Reduce()}$ is more efficient than any manual summing and use of
Send/Gather would be).

\todo[inline]{ANSWER QUESTION 6!!!!! (Check final section first!)}

\section{Program Load}

\subsection{Memory requirement}
\label{subsec:MemReq}

The program developed for this problem set requires a certain amount of
available memory to run. There are several factors and variables to take into
account, which can all differ according to the runmodes of the program (level of
parallelism or lack thereof) and circumstance.

Since the problem set specifies the circumstance $n \gg 1$, we will ignore the
negligible extra memory needed for OpenMP. This is because a set of threads
managed by OpenMP, specifically performing a summation (as shown in
subsection~\ref{lst:OpenMP}), will have an insignificant impact on the programs
memory usage.

Hence, below there are two equations where we detail how we believe the \textit{
non-negligible} memory requirement should behave when run in a single process,
or in $P$ processes with input size decided by variable $k$ as a deciding factor
in either case.

\begin{equation}
	noMPImem = supVars\footnote{supVars = support Variables. This amount is
static, and very negligible compared to the sum describing the size of the
vector.} + \sum_{i=1}^{2^k}
sizeof(double)
	\label{eq:noMPI-Mem}
\end{equation}

\begin{equation}
	MPImem = noMPImem + (P-1)\footnote{P = $2^m$ = amount of processes the
program is run on concurrently though MPI.} \cdot supVars + P \cdot
\sum_{i=1}^{2^{k-m}} sizeof(double)
	\label{eq:MPI-Mem}
\end{equation}

As detailed above in equation~\ref{eq:MPI-Mem} (parallel circumstance), you can
see that the memory requirement ``noMPImem'' detailed in
equation~\ref{eq:noMPI-Mem} (serial circumstance), will always be present.
However, the sum term in equation~\ref{eq:MPI-Mem} can be thought of as a
duplication of the sum representing the memory required for the length of the
vector representing the Hyperharmonic series in equation~\ref{eq:noMPI-Mem}.

The only difference being that its memory requirement is spread across the $P$
processes.

\subsection{Mathematical computation}

The computational load (mathematical complexity) of our program can be given as
the amount of \textit{\textbf{Fl}oating-point \textbf{Op}erations} (FLOPs) it
needs to complete for a given input. Since this program computes the sum of a
Hyperharmonic finite series, below we detail the math behind the programs
computations.

The mathematical operation needed to generate each term of the Hyperharmonic
series is as follows;

\begin{equation}
	v_i = \frac{1}{i^2}.
\end{equation}

Analyzing how many FLOPs are required for generating a term $v_i$, we see that
it requires a multiplication ($i\cdot i = i^2 = X$) and a division
($\frac{1}{X}$). A total of 2 FLOPs. So to generate a Hyperharmonic series of
$n$ terms, requires $2n$ FLOPs. Below we detail the formula used in the program
for this problem set when calculating the sum of our finite length Hyperharmonic
series.

The total amount of FLOPs needed for computing (\ref{eq:HyperSum}) will then be
$2n + n = 3n$. Thus, as we have used $k$ to define $n=2^k$ the computational
cost of our program can be described as; $2^k\cdot 3$ FLOPs.

\subsection{Load-balancing}

The program developed is very well load-balanced for this problem set. The
problem set specifies the length of the vectors (representing the Hyperharmonic
series) to be $2^k$ long, and that our MPI implementation should only run on a
system using $P$ processors where $P$ is a power of 2.

It should be mentioned that the program does some pre-processing serially,
mandated by the task and by MPI. The initialization is necessary and
unavoidable, and the generation of the vector would probably not benefit much
(if at all) by parallelization.

Either way, the negligible computations involved in the initialization of the
problem will be negilible compared to the overall  runtime for problem sizes of
any considerable size\footnote{$n \gg 1$, as  referenced in
subsection~\ref{subsec:MemReq}.}. This stands equally true for the  initial
generation of the vector representing the Hyperharmonic series of the problem
set. In any attempt at parallelizing the vector generation, the  necessary
overhead would dominate any gain.

Hence, the summation cost of the double-values in the vector of length $2^k$ can
always be divided equally among $P = 2^m$ processors, as long as $k\geq m$. The
parallelism is possible due to the fact that all partial sums are completely
independent of each other. \newline

Therefore, we believe that you can safely conclude that within the parameters
given in this problem set, this problem is very attractive to solve through
parallel processing.

\section{Results}

\todo[inline]{Implement this? I imagine this is where question 6 should be...}

\subsection{Serial implementation}

\subsection{OpenMP implementation}

\subsection{MPI implementation}

\subsection{MPI + OpenMP implementation}

\end{document}