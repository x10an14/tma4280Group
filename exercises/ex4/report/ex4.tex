%Exercise 4 LaTeX report for TMA 4280
\documentclass[fontsize=11pt,paper=a4,titlepage]{report}
% \usepackage{float} %dunno yet??, probably replaced by amsfonts
\usepackage{listings}
\usepackage{color}
\usepackage{amsfonts}	%For \mathbb
\usepackage{caption}	%Dunno yet
\usepackage{todonotes}	%For \todo

\lstset{ %
language=C,						% choose the language of the code
basicstyle=\footnotesize,		% the size of the fonts that are used for the code
numbers=left,					% where to put the line-numbers
numberstyle=\footnotesize,		% the size of the fonts that are used for the line-numbers
stepnumber=1,					% the step between two line-numbers. If it is 1 each line will be numbered
numbersep=5pt,					% how far the line-numbers are from the code
backgroundcolor=\color{white},	% choose the background color. You must add \usepackage{color}
showspaces=false,				% show spaces adding particular underscores
showstringspaces=false,			% underline spaces within strings
showtabs=false,					% show tabs within strings adding particular underscores
frame=single,					% adds a frame around the code
tabsize=4,						% sets default tabsize to 2 spaces
captionpos=b,					% sets the caption-position to bottom
breaklines=true,				% sets automatic line breaking
breakatwhitespace=false,		% sets if automatic breaks should only happen at whitespace
escapeinside={\%*}{*)}			% if you want to add a comment within your code
}

%% \usepackage{float} %dunno yet??, probably replaced by amsfonts
\usepackage{listings}
\usepackage[usenames,dvipsnames]{color}		%For the SkyBlue background color for lstlistings
\usepackage{amsfonts}	%For \mathbb
\usepackage{amsmath}
\usepackage{mathtools}
% \usepackage{caption}	%Dunno yet
\usepackage{todonotes}	%For \todo
\usepackage{tabularx}	%for tablecontents wrapping inside cell, instead of cell breaking page width.

\lstset{ %
language=C,							% choose the language of the code
basicstyle=\footnotesize,			% the size of the fonts that are used for the code
numbers=left,						% where to put the line-numbers
numberstyle=\footnotesize,			% the size of the fonts that are used for the line-numbers
stepnumber=1,						% the step between two line-numbers. If it is 1 each line will be numbered
numbersep=5pt,						% how far the line-numbers are from the code
backgroundcolor=\color{SkyBlue},	% choose the background color. You must add \usepackage{color}
showspaces=false,					% show spaces adding particular underscores
showstringspaces=false,				% underline spaces within strings
showtabs=false,						% show tabs within strings adding particular underscores
frame=single,						% adds a frame around the code
tabsize=4,							% sets default tabsize to 4 spaces
captionpos=b,						% sets the caption-position to bottom
breaklines=true,					% sets automatic line breaking
breakatwhitespace=false,			% sets if automatic breaks should only happen at whitespace
escapeinside={\%*}{*)}				% if you want to add a comment within your code
}

\lstset{literate=
	{á}{{\'a}}1 {é}{{\'e}}1 {í}{{\'i}}1 {ó}{{\'o}}1 {ú}{{\'u}}1
	{Á}{{\'A}}1 {É}{{\'E}}1 {Í}{{\'I}}1 {Ó}{{\'O}}1 {Ú}{{\'U}}1
	{à}{{\`a}}1 {è}{{\'e}}1 {ì}{{\`i}}1 {ò}{{\`o}}1 {ù}{{\`u}}1
	{À}{{\`A}}1 {È}{{\'E}}1 {Ì}{{\`I}}1 {Ò}{{\`O}}1 {Ù}{{\`U}}1
	{ä}{{\"a}}1 {ë}{{\"e}}1 {ï}{{\"i}}1 {ö}{{\"o}}1 {ü}{{\"u}}1
	{Ä}{{\"A}}1 {Ë}{{\"E}}1 {Ï}{{\"I}}1 {Ö}{{\"O}}1 {Ü}{{\"U}}1
	{â}{{\^a}}1 {ê}{{\^e}}1 {î}{{\^i}}1 {ô}{{\^o}}1 {û}{{\^u}}1
	{Â}{{\^A}}1 {Ê}{{\^E}}1 {Î}{{\^I}}1 {Ô}{{\^O}}1 {Û}{{\^U}}1
	{œ}{{\oe}}1 {Œ}{{\OE}}1 {æ}{{\ae}}1 {Æ}{{\AE}}1 {ß}{{\ss}}1
	{ç}{{\c c}}1 {Ç}{{\c C}}1 {ø}{{\o}}1 {å}{{\r a}}1 {Å}{{\r A}}1
	{€}{{\EUR}}1 {£}{{\pounds}}1
}
 %config.tex file in same directory

\title{Problem set 4}
\author{Christian CHAVEZ, Mireia DUASO, Erlend SIGHOLT\\
TMA4280 Introduction to supercomputing}

\begin{document}

\maketitle

\abstract{

\todo[inline]{EMPTY!}

}
%%%% 		INTRODUCTION			%%%

%\chapters{}
\section{Introduction}
For the programming belonging to this problem set, see $\textit{ex4.c}$ file in
the zipped archive attached to this report. This program calculates the
difference between the sum of a finite hyperharmonic series, and the convergence
sum for an infinite hyperharmonic series. This program is able to compute the
difference between the two in the following scenarioes;

\begin{enumerate}
	\item{Using one thread and a single process on a single processing core
for the program's computations.}
	\item{Using one process and multiple threads concurrently on a single
processing core for the program's computations.}
	\item{Using multiple processing cores concurrently for the program's
computations.}
	\item{Using multiple threads on multiple processing cores concurrently for
the program's computations.}
	\label{RunMode}
\end{enumerate}
\todo{FIX CAPTION!}

\section{Series $S_n$}

A \textit{hyperharmonic} series is a series of the following form;

\begin{displaymath}
	\sum_{i=1}^{n} \frac{1}{i^p}, \quad \quad with \quad n = \infty.
	\ref{HyperSeries}
\end{displaymath}

This problem set wants us to write a program with $n = 2^k$, where $k$ is given
as parameter at \todo{Confirm phrasing of program initiation}program start. This
problem set also specifies $p$ to be a constant; $p = 2$. With $p = 2$, the sum
of this hyperharmonic series converges to $\frac{\pi^2}{6}$.

The objective of the problem set is to compute the difference between
$\frac{\pi^2}{6}$ and the sum of a finite series where $n = 2^k$, with $\{k \in
\mathbb{N} : k \in [3, 14]\}$. This difference is what the program written for
this problem set is supposed to calculate.

\section{Concurrency Implementation}

The program was developed in iterations. The first iteration ran on a single
thread in a single process on a single processor core, in a non-concurrent
sequential fashion. The second iteration introduced running the program as
mentioned in item 2 in \cite{RunMode}. This was enabled by putting the following
line into our \textit{ex4.c} file.

% \begin{lslisting}
% 	#pragma omp parallel for schedule(dynamic, 5) reduction(+:sum)
% \end{lslisting}

The third iteration focused on combining items 3 and 4 from \cite{RunMode}.
Since OpenMP is enabled through the use of pragmas, making the code run with MPI
was then only a matter of choosing the right compiler and compilation switches.
Hence, in effect the third iteration only focused on implementing MPI into the
program.

To utilize MPI, the program does need to use a subset of the available MPI
function calls; $\textit{MPI\_Scatter()}$, $\textit{MPI\_Send()}$,
$\textit{MPI\_Receive()}$, $\textit{MPI\_Gather()}$, $\textit{MPI\_Init()}$,
$\textit{MPI\_Finalize()}$, and so on. Not all of these are necessary, but a
minimum subset of these are. Such as, $\textit{MPI\_Init()}$ and $\textit{
MPI\_Finalize()}$, and then one of the $\textit{MPI\_Scatter()}$ alternatives ,
or $\textit{MPI\_Send()}$. Likewise there are more than one
$\textit{MPI\_Gather()}$, but only one of its forms or $\textit{MPI\_Receive()}$
are necessary.

\todo[inline]{ANSWER QUESTION 6!!!!!}

\todo[inline]{ANSWER QUESTION 7!!!!!}

\todo[inline]{ANSWER QUESTION 8!!!!!}

\todo[inline]{ANSWER QUESTION 9!!!!!}

% \input{chapters/implementation}
% \input{chapters/methodology}
% \input{chapters/results}
% \input{chapters/conclusion}

%\lstlistoflistings
\listoffigures
\listoftables

\end{document}