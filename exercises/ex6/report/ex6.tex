%Exercise 4 LaTeX report for TMA 4280
\documentclass[fontsize=11pt,paper=a4,titlepage]{report}
% \usepackage{float} %dunno yet??, probably replaced by amsfonts
\usepackage{listings}
\usepackage[usenames,dvipsnames]{color}		%For the SkyBlue background color for lstlistings
\usepackage{mathtools}
\usepackage{amsfonts,amsmath,amssymb,amsthm}	%For \mathbb
% \usepackage{caption}	%Dunno yet
\usepackage{todonotes}	%For \todo
\usepackage{tabularx}	%for tablecontents wrapping inside cell, instead of cell breaking page width.
\usepackage{verbatim}
\usepackage[margin=3cm]{geometry}

\newcommand*\Laplace{\mathop{}\!\mathbin\bigtriangleup}

\lstset{ %
language=C,							% choose the language of the code
basicstyle=\footnotesize,			% the size of the fonts that are used for the code
numbers=left,						% where to put the line-numbers
numberstyle=\footnotesize,			% the size of the fonts that are used for the line-numbers
stepnumber=1,						% the step between two line-numbers. If it is 1 each line will be numbered
numbersep=5pt,						% how far the line-numbers are from the code
backgroundcolor=\color{SkyBlue},	% choose the background color. You must add \usepackage{color}
showspaces=false,					% show spaces adding particular underscores
showstringspaces=false,				% underline spaces within strings
showtabs=false,						% show tabs within strings adding particular underscores
frame=single,						% adds a frame around the code
tabsize=4,							% sets default tabsize to 4 spaces
captionpos=b,						% sets the caption-position to bottom
breaklines=true,					% sets automatic line breaking
breakatwhitespace=false,			% sets if automatic breaks should only happen at whitespace
escapeinside={\%*}{*)}				% if you want to add a comment within your code
}

\lstset{literate=
	{á}{{\'a}}1 {é}{{\'e}}1 {í}{{\'i}}1 {ó}{{\'o}}1 {ú}{{\'u}}1
	{Á}{{\'A}}1 {É}{{\'E}}1 {Í}{{\'I}}1 {Ó}{{\'O}}1 {Ú}{{\'U}}1
	{à}{{\`a}}1 {è}{{\'e}}1 {ì}{{\`i}}1 {ò}{{\`o}}1 {ù}{{\`u}}1
	{À}{{\`A}}1 {È}{{\'E}}1 {Ì}{{\`I}}1 {Ò}{{\`O}}1 {Ù}{{\`U}}1
	{ä}{{\"a}}1 {ë}{{\"e}}1 {ï}{{\"i}}1 {ö}{{\"o}}1 {ü}{{\"u}}1
	{Ä}{{\"A}}1 {Ë}{{\"E}}1 {Ï}{{\"I}}1 {Ö}{{\"O}}1 {Ü}{{\"U}}1
	{â}{{\^a}}1 {ê}{{\^e}}1 {î}{{\^i}}1 {ô}{{\^o}}1 {û}{{\^u}}1
	{Â}{{\^A}}1 {Ê}{{\^E}}1 {Î}{{\^I}}1 {Ô}{{\^O}}1 {Û}{{\^U}}1
	{œ}{{\oe}}1 {Œ}{{\OE}}1 {æ}{{\ae}}1 {Æ}{{\AE}}1 {ß}{{\ss}}1
	{ç}{{\c c}}1 {Ç}{{\c C}}1 {ø}{{\o}}1 {å}{{\r a}}1 {Å}{{\r A}}1
	{€}{{\EUR}}1 {£}{{\pounds}}1
}
 %config.tex file in same directory

\begin{document}

\begin{center}

%\lstlistoflistings
% \listoffigures
% \listoftables

{\huge Problem set 6}\\[0.5cm]

\textsc{\LARGE TMA4280}\\[0.5cm]
\textsc{\large Introduction to supercomputing -}\\
\textsc{\large Mandatory problem set}\\[0.6cm]

\begin{table}[h]
\centering
\begin{tabular}{ccc}
	\textsc{Christian CHAVEZ} & \textsc{Mireia DUASO} & \textsc{Erlend SIGHOLT}
\end{tabular}
\end{table}

\large{\today}
\vfill
\section*{Abstract}
\end{center}

\todo[inline]{ABSTRACT}

\addtocounter{chapter}{1}

\clearpage
\section{Introduction}

For the programming code belonging to this problem set, see the $\textit{ex6.c}$
file in the zipped archive attached to this report. This program calculates the
solution of the two-dimensional Poisson problem

\begin{eqnarray}
	-\nabla^2 u & = & f \quad \textrm{in} \quad \Omega \\
	u & = & 0 \quad \textrm{on} \quad \partial\Omega
	\label{eq:Poisson}
\end{eqnarray}

considering different sources, boundary conditions and domains.

\todo[inline]{Explain the first points, the code and the results.}

\todo[inline]{When we comment the code we have to say the error we made during
the execution of the program.}

\section{Modifying the original problem}

\todo{Dealing with complexities?}

% (f)

We want to know which part of our code has to be modified if we change the 
original problem. First of all, we want to deal with the case $u \neq 0$ on 
$\partial\Omega$, where $\Omega = (0,1) \times (0,1)$. To tackle the new problem
we define $v$ as a new solution,

\begin{equation}
	v = u - g
\end{equation}

for some lifting function $g$. This $g$ is supposed to make the new defined 
solution $0$ in the boundary condition.
This is in the analytic case. If we want to apply this theory in our method we
are going to transform the original problem like

\begin{displaymath}
	\mathbf{A} \times (\mathbf{u} + \mathbf{u_B}) = h^2 \mathbf{f}
\end{displaymath}

where $\mathbf{A}$ will be a $(n + 1) \times (n + 1)$ matrix, $\mathbf{u}$ will 
be the old solution, $\mathbf{f}$ the source, $h$ the length of the step and 
$\mathbf{u_B}$ is the function that satisfies the boundary conditions.

We had the next system

\begin{displaymath}
\begin{bmatrix}
	2 & -1 &  &  &  &  &  \\
	-1 & 2 & -1 &  &  &  &  \\
	 & -1 & 2 &  &  &  &  \\
	 &  & \ddots & \ddots & \ddots &  & \\
	 &  &  &  & 2 & -1 &  \\
	 &  &  &  & -1 & 2 & -1 \\
	 &  &  &  &  & -1 & 2
\end{bmatrix}
\begin{bmatrix*}[c]
	u_0 \\
	u_1 \\
	u_2 \\
	\vdots \\
	u_{n - 2} \\
	u_{n - 1} \\
	u_n
\end{bmatrix*}
=
\begin{bmatrix*}[c]
	f_0 \\
	f_1 \\
	f_2 \\
	\vdots \\
	f_{n - 2} \\
	f_{n - 1} \\
	f_n
\end{bmatrix*}.
\end{displaymath}

Since we had homogeneous boundary conditions we reduced the problem $2$ 
dimensions to become an $(n - 1) \times (n - 1)$ system because $u_0 = u_n = 0$ 
but now we have values for the boundaries so, if we compute the left side 
operation, we get

\begin{displaymath}
\begin{bmatrix}
	2u_0 - u_1 \\
	- u_0 + 2u_1 - u_2 \\
	- u_1 + 2u_2 - u_3 \\
	\vdots \\
	- u_{n - 1} + 2u_{n - 2} - u_{n - 1} \\
	- u_{n - 2} + 2u_{n - 1} - u_{n} \\
	- u_{n - 1} + 2u_n
\end{bmatrix}
=
\begin{bmatrix*}[c]
	f_0 \\
	f_1 \\
	f_2 \\
	\vdots \\
	f_{n - 2} \\
	f_{n - 1} \\
	f_n
\end{bmatrix*}.
\end{displaymath}

but this system is redundant because there are $n - 1$ variables but $n + 1$ 
euqation so we can remove the first and the last term and we just need to plus a
vector to our old matrix. And the system is going to be written as

\begin{displaymath}
\begin{bmatrix}
	2 & -1 &  &  &  &  &  \\
	-1 & 2 & -1 &  &  &  &  \\
	 & -1 & 2 &  &  &  &  \\
	 &  & \ddots & \ddots & \ddots &  & \\
	 &  &  &  & 2 & -1 &  \\
	 &  &  &  & -1 & 2 & -1 \\
	 &  &  &  &  & -1 & 2
\end{bmatrix}
\begin{bmatrix*}[c]
	u_1 \\
	u_2 \\
	\vdots \\
	u_{n - 2} \\
	u_{n - 1}
\end{bmatrix*}
+ \begin{bmatrix*}[c]
	- u_0 \\
	0 \\
	\vdots \\
	0 \\
	- u_n
\end{bmatrix*}
=
\begin{bmatrix*}[c]
	f_1 \\
	f_2 \\
	\vdots \\
	f_{n - 2} \\
	f_{n - 1}
\end{bmatrix*}.
\end{displaymath}

\todo[inline]{Check with the previous points}

%(g)

\todo{Compare with the implementation!}

We want to think in the geometry of the domain. We want to solve the
PDE~\ref{eq:Poisson} in a rectangle $\Omega = (0, L_x) \times (0, L_y)$. To do
that we have to declare two length instead of our old $h$, $h_x$ and $h_y$,
which they will have the same value if we work in a square and different values
if we work in a rectangle.

Following the diagonalisation done in the lectures the only change it has to be
implemented is, instead of clear the $\mathbf{A}$ up and put the $h^2$ with the 
$\mathbf{f}$, 
keeping the values $h_x^2$ and $h_y^2$ in $\mathbf{f}$. So, when we go through 
the diagonalisation method we will devide de eigenvalues by $h_x^2 + h_y^2$.

\todo[inline]{En vez de despejar la h y ponerla en la f, tenemos que dividirla 
en los valores propios... Estudiar bien la diagonalización de la matriz con los 
valores y los vectores propios.}

\end{document}
