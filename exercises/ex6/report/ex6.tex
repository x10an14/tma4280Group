%Exercise 4 LaTeX report for TMA 4280
\documentclass[fontsize=11pt,paper=a4,titlepage]{article}
% \usepackage{float} %dunno yet??, probably replaced by amsfonts
\usepackage{listings}
\usepackage[usenames,dvipsnames]{color}		%For the SkyBlue background color for lstlistings
\usepackage{mathtools}
\usepackage{amsfonts,amsmath,amssymb,amsthm}	%For \mathbb
% \usepackage{caption}	%Dunno yet
\usepackage{todonotes}	%For \todo
\usepackage{tabularx}	%for tablecontents wrapping inside cell, instead of cell breaking page width.
\usepackage{verbatim}
\usepackage[margin=3cm]{geometry}

\newcommand*\Laplace{\mathop{}\!\mathbin\bigtriangleup}

\lstset{ %
language=C,							% choose the language of the code
basicstyle=\footnotesize,			% the size of the fonts that are used for the code
numbers=left,						% where to put the line-numbers
numberstyle=\footnotesize,			% the size of the fonts that are used for the line-numbers
stepnumber=1,						% the step between two line-numbers. If it is 1 each line will be numbered
numbersep=5pt,						% how far the line-numbers are from the code
backgroundcolor=\color{SkyBlue},	% choose the background color. You must add \usepackage{color}
showspaces=false,					% show spaces adding particular underscores
showstringspaces=false,				% underline spaces within strings
showtabs=false,						% show tabs within strings adding particular underscores
frame=single,						% adds a frame around the code
tabsize=4,							% sets default tabsize to 4 spaces
captionpos=b,						% sets the caption-position to bottom
breaklines=true,					% sets automatic line breaking
breakatwhitespace=false,			% sets if automatic breaks should only happen at whitespace
escapeinside={\%*}{*)}				% if you want to add a comment within your code
}

\lstset{literate=
	{á}{{\'a}}1 {é}{{\'e}}1 {í}{{\'i}}1 {ó}{{\'o}}1 {ú}{{\'u}}1
	{Á}{{\'A}}1 {É}{{\'E}}1 {Í}{{\'I}}1 {Ó}{{\'O}}1 {Ú}{{\'U}}1
	{à}{{\`a}}1 {è}{{\'e}}1 {ì}{{\`i}}1 {ò}{{\`o}}1 {ù}{{\`u}}1
	{À}{{\`A}}1 {È}{{\'E}}1 {Ì}{{\`I}}1 {Ò}{{\`O}}1 {Ù}{{\`U}}1
	{ä}{{\"a}}1 {ë}{{\"e}}1 {ï}{{\"i}}1 {ö}{{\"o}}1 {ü}{{\"u}}1
	{Ä}{{\"A}}1 {Ë}{{\"E}}1 {Ï}{{\"I}}1 {Ö}{{\"O}}1 {Ü}{{\"U}}1
	{â}{{\^a}}1 {ê}{{\^e}}1 {î}{{\^i}}1 {ô}{{\^o}}1 {û}{{\^u}}1
	{Â}{{\^A}}1 {Ê}{{\^E}}1 {Î}{{\^I}}1 {Ô}{{\^O}}1 {Û}{{\^U}}1
	{œ}{{\oe}}1 {Œ}{{\OE}}1 {æ}{{\ae}}1 {Æ}{{\AE}}1 {ß}{{\ss}}1
	{ç}{{\c c}}1 {Ç}{{\c C}}1 {ø}{{\o}}1 {å}{{\r a}}1 {Å}{{\r A}}1
	{€}{{\EUR}}1 {£}{{\pounds}}1
}
 %config.tex file in same directory

\begin{document}

\begin{center}

% \lstlistoflistings
% \listoffigures
% \listoftables

{\huge Problem set 6}\\[0.5cm]

\textsc{\LARGE TMA4280}\\[0.5cm]
\textsc{\large Introduction to supercomputing -}\\
\textsc{\large Mandatory problem set}\\[0.6cm]

\begin{table}[h]
\centering
\begin{tabular}{ccc}
	\textsc{Christian CHAVEZ} & \textsc{Mireia DUASO} & \textsc{Erlend SIGHOLT}
\end{tabular}
\end{table}

\large{\today}
\vfill
% \section*{Abstract}
\end{center}

\todo[inline]{ABSTRACT}


\clearpage
\section{Introduction}

For the programming code belonging to this problem set, see the $\textit{ex6.c}$
and $\textit{ex6.h}$ files in the zipped archive attached to this hand-in.

For this problem set, we have written our solution in C inspired by the work of
our professor in the subject this problem set belongs to. Hence, our structures,
the general way of thinking, and the algorithms used are based on the work and
lectures done in this subject~\cite{tma4280}.


\section{The Poisson problem}
\label{sec:Pois-Prob}

Poisson's equation is an elliptic Partial Differential Equation (PDE) which is
used to model diffusion. It appears in electrostatics, mechanical
engineering, and theoretical physics, among many others.

The problem follows the expression

\begin{eqnarray}
	-\nabla^2 u & = & f \quad \textrm{in} \quad \Omega \\
	u & = & g \quad \textrm{on} \quad \partial\Omega
	\label{eq:Poisson}
\end{eqnarray}

where $f$ is a source function, $g$ is the boundary condition, and $\Omega$ is a
bounded domain.

And as such, a parallelized solver for 2-Dimensional Poisson PDE's is the
objective of this problem set. It's our implementation of this solver we
describe and analyze in this report.

\section{The Problem Discretized}
\label{sec:Prob-Discr}

We have the below system.

\begin{displaymath}
\begin{bmatrix}
	2 & -1 &  &  &  &  &  \\
	-1 & 2 & -1 &  &  &  &  \\
	 & -1 & 2 & -1 &  &  &  \\
	 &  & \ddots & \ddots & \ddots &  & \\
	 &  &  & -1 & 2 & -1 &  \\
	 &  &  &  & -1 & 2 & -1 \\
	 &  &  &  &  & -1 & 2
\end{bmatrix}
\begin{bmatrix*}[c]
	u_0 \\
	u_1 \\
	u_2 \\
	\vdots \\
	u_{n - 2} \\
	u_{n - 1} \\
	u_n
\end{bmatrix*}
=
\begin{bmatrix*}[c]
	f_0 \\
	f_1 \\
	f_2 \\
	\vdots \\
	f_{n - 2} \\
	f_{n - 1} \\
	f_n
\end{bmatrix*}
\end{displaymath}

Since we have homogeneous boundary conditions, we reduce the problem by $2$
dimensions to become an $(n - 1) \times (n - 1)$ system since $u_0 = u_n = 0$.

\section{The Implemented Solver}

This section is divided into four parts; first we describe our implementation
and analyze its complexity, then we explain how we enabled it to accept smooth
functions, before we finish this section by explaining how we confirmed that our
solver converges towards the correct answer for a given smooth function.

\subsection{Our Implementation}
\label{sec:Impl}

Our implementation is based on the fact and observation that for any Symmetric
Positive Definite (SPD) matrix, we can through the use of a Fast Fourier
Transform (FFT), more specifically the the Discrete Sine Transform (DST)
algorithm, implement a solution whose algorithmic FLOP complexity should be
$\mathcal{O}(n\times log(n))$.

Therefore, we will below first describe our implementations dataflow and way
``way of thinking'', before analyzing its final FLOP and memory complexity.

\subsubsection{Implementation Dataflow}

Our implementation is based on the observation that the following four steps
need to happen in the solver for it to function correctly.

\begin{enumerate}
	\item \label{impl-step-1} Generate the $(N - 1)^2$ unknown variables across
	all processes, by filling each process $\approx \frac{N-1}{Amount of
	processes}$, and fill each variable with values corresponding to the
	function we want to perform the 2-Dimensional Poisson operation on.

	\item \label{impl-step-2} Use the Fourier Sine Transform (FST) on each
	coloumn of the matrix within each process, before transposing the whole
	matrix across all processes, requiring one ``round of communication'' across
	all processes, before performing an inverse FST on the now transposed matrix
	to work with the inverted matrix.

	\item \label{impl-step-3} Perform the tensor operation based on the the
	five-point stencile given from the 2-Dimensional nature of the operation on
	each element by dividing $a_{xy}$ on $d_x + d_y$ where $d$ is a vector
	(list) containing the eigenvalues for these types of matrices.

	\item \label{impl-step-4} Repeat step~\ref{impl-step-2}, except now perform
	the FST on the transposed matrix, transpose it back, and perform the inverse
	FST on the un-transposed matrix.

	%\label{list:impl-steps}
\end{enumerate}

Test item in enumerate ref:~\ref{impl-step-2}.

\subsubsection{Implementation Complexity}


%section e)
\subsection{Accepting Smooth Functions}

\subsection{Correctness Convergence}

\todo[inline]{Decide if we keep this section, to analyze and report communication/FLOP, FLOPS, and so on...}

\section{Results}

\subsection{MPI Results}

\subsection{OpenMP Results}

\subsection{Combined Results}

% (f)
\section{Non-Homogenous Dirichlet Boundary Conditions}

We want to know which part of our code has to be modified if we change the
original problem, when $u \neq 0$ on $\partial\Omega$, and $\Omega = (0,1)
\times (0,1)$.

In the analytic case, to tackle the new problem we define $v$ as a new solution,

\begin{equation}
	v = u - g
\end{equation}

for some lifting function $g$. In the newly defined solution, $g$ is supposed to
set the values of $v$ along the boundaries to $0$.

If we want our code to solve the problem with these new conditions, we need to
tranform the original problem into

\begin{displaymath}
	\mathbf{A} \times (\mathbf{u} + \mathbf{u_B}) = h^2 \mathbf{f}
\end{displaymath}

where $\mathbf{A}$ will be a $(n + 1) \times (n + 1)$ matrix, $\mathbf{u}$ will
be the old solution, $\mathbf{f}$ the source, $h$ the length of the step, and
$\mathbf{u_B}$ the function that satisfies the boundary conditions.

Considering the discretization of the original problem in section~\ref{sec:Prob-Discr}, if we re-compute the left hand-side of that system, we get

\begin{displaymath}
\begin{bmatrix}
	2u_0 - u_1 \\
	- u_0 + 2u_1 - u_2 \\
	- u_1 + 2u_2 - u_3 \\
	\vdots \\
	- u_{n - 3} + 2u_{n - 2} - u_{n - 1} \\
	- u_{n - 2} + 2u_{n - 1} - u_{n} \\
	- u_{n - 1} + 2u_n
\end{bmatrix}
=
\begin{bmatrix*}[c]
	f_0 \\
	f_1 \\
	f_2 \\
	\vdots \\
	f_{n - 2} \\
	f_{n - 1} \\
	f_n
\end{bmatrix*}.
\end{displaymath}

but this system is redundant since there are $n - 1$ variables, and $n + 1$
euqations. Hence, we can remove the first and the last terms, and we only need
to add a vector to our previous matrix.

Then the system will be written as

\begin{displaymath}
\begin{bmatrix}
	2 & -1 &  &  &  &  &  \\
	-1 & 2 & -1 &  &  &  &  \\
	 & -1 & 2 &  &  &  &  \\
	 &  & \ddots & \ddots & \ddots &  & \\
	 &  &  &  & 2 & -1 &  \\
	 &  &  &  & -1 & 2 & -1 \\
	 &  &  &  &  & -1 & 2
\end{bmatrix}
\begin{bmatrix*}[c]
	u_1 \\
	u_2 \\
	\vdots \\
	u_{n - 2} \\
	u_{n - 1}
\end{bmatrix*}
+ \begin{bmatrix*}[c]
	- u_0 \\
	0 \\
	\vdots \\
	0 \\
	- u_n
\end{bmatrix*}
=
\begin{bmatrix*}[c]
	f_1 \\
	f_2 \\
	\vdots \\
	f_{n - 2} \\
	f_{n - 1}
\end{bmatrix*}.
\end{displaymath}

\section{Handling Different Domains}
%(g)

If we consider the geometry of the domain, we may want to solve the
PDE~\ref{eq:Poisson} in a rectangle $\Omega = (0, L_x) \times (0, L_y)$. To
accomplish this, we need to declare two lengths instead of our old $h$. Namely,
$h_x$ and $h_y$, which would have the same value if we considered a square, and
different values than we would get working with a rectangle.

Following the diagonalization shown in the lectures, the only change that needs
to be implemented is the following.

In the implemented tensor-operation, instead of simply summing the two
eigenvalues whose sum we divide the variable with, we need to divide each
eigenvalue with their corresponding step-lengths of the finite grid, $h_x^2$ and
$h_y^2$. This is due to the fact that we do not store our step-lengths in the
source function.

\section{Conclusion}

\todo[inline]{Summarize the best/worst results of the result chapter. Also
repeat what we can and cannot to do from Mireia's math-section.}

\bibliography{bibliography}{}
\bibliographystyle{plain}

\end{document}
