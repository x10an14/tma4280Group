\documentclass{beamer}
\usepackage{amsmath}
\usepackage{rotating}
\usepackage{graphicx}
\usepackage{multimedia}
\usepackage{listings}

\useinnertheme[shadow=true]{rounded}
\useoutertheme{shadow}
\usecolortheme{orchid}
\usecolortheme{whale}

\mode<presentation>

\newcommand{\dif}{\, \mathrm{d}}
\newcommand{\diff}[2]{\frac{\mathrm{d}#1}{\mathrm{d}#2}}
\newcommand{\partdiff}[2]{\frac{\partial #1}{\partial #2}}


\title{TMA4280 - Introduction to supercomputing}
\subtitle{Kongull crash course}
\author{Arne Morten Kvarving}
\institute{NTNU and SINTEF ICT}
\date{February 2012}

\begin{document}

\maketitle
\begin{frame}\frametitle{Supercomputing - a users guide}
\begin{itemize}
    \item Supercomputers use, with very few exceptions, a unix-type operating system.
    \item Predominantly Linux these days.
    \item Varying degree of unix knowledge.
    \item This will be a crash course in using one such systems (kongull).
\end{itemize}
\end{frame}
\begin{frame}\frametitle{How do I log onto the system?}
\begin{itemize}
    \item Login is handled through \emph{secure shell} or \texttt{ssh} for short.
    \item On a Linux or OSX computer, this is built into the operating system.
    \item On Windows you need to use special software called a SSH client.
    \item One example of a client: Putty.
\end{itemize}
\end{frame}
\begin{frame}\frametitle{How do I log onto the system?}
\begin{itemize}
    \item We will be using the \texttt{kongull} cluster.
    \item URL: kongull.hpc.ntnu.no
    \item On a Linux/OSX shell: \small \lstinputlisting{login} \normalsize
    \item Note: This machine is NOT accessible from the outside world.
\end{itemize}
\end{frame}
\begin{frame}\frametitle{How do I transfer files to/from the system?}
\begin{itemize}
    \item File transfers is performed using \emph{secure copy} or \texttt{scp} for short.
    \item On a Linux or OSX computer, this is built into the operating system.
    \item On Windows you need to use special software capable of performing SSH transfers.
    \item One example of such an application is \emph{WinSCP}.
    \item Alternative approach Linux: \texttt{sshfs}.
    \item Alternative approach Windows: \texttt{mindTerm/NetDrive}.
\end{itemize}
\end{frame}
\begin{frame}\frametitle{Damn password}
\begin{itemize}
    \item You can avoid it: \lstinputlisting{keys}
    \item Putty has some kind of key mechanism but I do not know the details.
\end{itemize}
\end{frame}
\begin{frame}\frametitle{Damn host name, so long}
\begin{itemize}
    \item You can setup shortcuts for ssh. \$HOME/.ssh/config
        \lstinputlisting{sshconfig}
\end{itemize}
\end{frame}
\begin{frame}\frametitle{Editing files}
\begin{itemize}
    \item Two choices: text mode editor or graphical editor.
    \item Text mode editors: emacs, vim, gedit, nano.
\end{itemize}
\end{frame}
\begin{frame}\frametitle{Graphical display - X11 forwarding}
\begin{itemize}
    \item The graphical windowing system on unix is called X11.
    \item This can be \emph{tunneled} over ssh.
    \item This way, an application can run on one computer and display
        on another.
    \item For Linux: \lstinputlisting{sshx}
    \item For OSX: Start X11.app, then it's the same.
    \item For windows: X-Win32. Available on progdist.
\end{itemize}
\end{frame}
\begin{frame}\frametitle{The modules system}
\begin{itemize}
    \item All software on NOTUR systems are offered through the modules system.
    \item Relevant commands: \lstinputlisting{modules}
    \item Relevant modules: intelcomp, openmpi, cmake, mercurial
\end{itemize}
\end{frame}
\begin{frame}\frametitle{Using cmake}
\begin{itemize}
  \item You need to tell cmake to use the intel compilers:
    \lstinputlisting{cmake}
\end{itemize}
\end{frame}
\begin{frame}\frametitle{Running jobs throug the queue system}
\begin{itemize}
   \item You have to write a job script.
     \lstinputlisting[basicstyle=\tiny]{jobscript}
   \item Note that we have to load (some of) the modules in the script as well!
\end{itemize}
\end{frame}
\begin{frame}\frametitle{Running jobs throug the queue system}
\begin{itemize}
   \item You have to make sure it is marked as executable:
     \lstinputlisting{chmod}
   \item You submit to the queue through:
     \lstinputlisting{qsub}
   \item You see the progress of your job through:
     \lstinputlisting{showq}
   \item You see the output of your program in files:
     \lstinputlisting{listing}
\end{itemize}
\end{frame}
\end{document}
