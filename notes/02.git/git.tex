\documentclass[twoside, 11pt, a4paper]{article}
\usepackage[latin1]{inputenc}
\usepackage[english]{babel}
\usepackage{amsmath,amsfonts,amssymb,graphicx,parskip}
\usepackage[T1]{fontenc}
%\usepackage{fourierx} % eller lmodern
\usepackage{subfigure}
\usepackage{multirow}
\usepackage{float}
\usepackage{array}
\restylefloat{figure}
\usepackage{listings}

% for debug utskrift
\newcommand{\debug}[1]{\texttt{#1}}
% for ingen utskrift av kommentarer/debug
%\newcommand{\debug}[1]{}

\DeclareMathOperator{\erf}{erf}
\DeclareMathOperator{\erfc}{erfc}
\DeclareMathOperator{\eps}{\epsilon}
\newcommand{\dee}{\mathrm{d}}

\begin{document}
\LARGE
\begin{center}
TMA4280: Introduction to Supercomputing
\end{center}
\vspace{1in}

\begin{center}
{Using GIT to syncronize code between multiple machines}
\end{center}

\Large
\vspace{0.5in}
\begin{center}
Spring 2014
\end{center}

\vspace{0.5in}

\large

\newpage
The scope of this document is to explain the basic mechanisms of GIT.
GIT is a complex tool, using it to its full power can take quite some 
time to learn. Something crucial may have been missed while 
attempting to boil it down to basics. There is a whole internet full of guides
out there and you are encouraged to supplement these ramblings with more verbose
tutorials and articles. Also, please do not hesitate to ask during lectures
or breaks; I love it when people talk GIT to me.

\subsection*{Installing GIT}
Start by installing git on your laptop; on ubuntu you can do this through
\emph{sudo apt-get install git}.

\subsection*{Setting up a GIT repository}
You can set up a GIT directory in two ways. You either clone a remote
repository, or by initializing a new empty GIT.
In either case you end up on a \emph{branch} called \emph{master} by default.

\subsubsection*{Cloning a remote repository}
You do this through \\
\begin{center}\emph{git clone <url>}\end{center}
Here <url> can be a local directory or a remote directory.

\subsubsection*{Setting up a new GIT}
Change to the directory you want to set up the GIT in and type \\
\begin{center}\emph{git init .}\end{center}
As far as GIT is concerned, adding files already existing 
locally is handled as a change to the file where everything was changed
(see further down). You just want to add source code, not files generated
by the build system or the compiler such as object files, libraries,
executables or scripts.

\subsection*{Layout of a GIT}
A GIT is divided in two parts; the repository information and your
local working tree. The first contains information about all the individual
commits, commit message, and branches (a sequence of commits) that is
recorded. This is stored in a folder called \emph{.git} in the root of the
repository. The second part is your local copy of the files in the GIT.

You can make a \emph{bare} clone of a GIT. This is a directory that only
contains the repository information, i.e. only the \emph{.git} folder of a
normal clone, and no local working tree. You do this through
\begin{center}\emph{git clone $--bare$ <url>}\end{center}
You can also initialize a bare, empty repository using
\begin{center}\emph{git init $--bare$ .}\end{center}

\subsection*{Keeping track of changes}
You can see what changes you have made to your local
working directory through
\begin{center}\emph{git diff}\end{center}
If you just want to see which files, but not the changes themself
you can use
\begin{center}\emph{git status}\end{center}
This is in general quite useful, it can show more things such
as which changes are marked for commit.

\subsubsection*{Marking my changes for commit}
\begin{center}\emph{git add <file>}\end{center}
This marks the change to these files for commit, 
but they are not yet recorded in the GIT. To do that we have
to \emph{commit} the changes.
\subsubsection*{Commiting changes to a GIT}
You commit changes to the GIT through \\
\begin{center}\emph{git commit}\end{center}
This will open a text editor where you can enter a commit message.
You can control which editor by adding \\
\begin{center}\emph{export EDITOR=myfav}\end{center}
to your \emph{$\sim$/.bashrc} file. Alternatively, you can specify the
commit message on the command line by doing
\begin{center}\emph{git commit -m "my message"}\end{center}
You can quickly commit all changes to all files tracked by
the GIT through \\
\begin{center}\emph{git commit -a}\end{center}
The two can be combined.
\subsection*{Keeping track of commits}
You can see the log of commits through
\begin{center}\emph{git log}\end{center}
You can the changes in an individual commit through
\begin{center}\emph{git show <revision>}\end{center}
The revision can be found using git log. Alternatively, you
have a few shortcuts. If you omit the revision, the last commit
in the branch will be shown. If you want to show the previous commit,
you can use
\begin{center}\emph{git show HEAD$\sim$1}\end{center}
The number can be increased.

\subsection*{Working with remote repositories}
A remote repository is a clone of this repository. Since GIT is
a distributed revision control system, you can clone a clone, and
it still contains all the information.

\subsubsection*{Adding a remote repository}
You tell GIT where the remote repository is located through \\
\begin{center}\emph{git remote add <name> <url>}\end{center}
If you initialize a repository through cloning another, the
repository you cloned will be registered as a remote named \emph{origin}.
If you started from scratch, there will be no remotes by default.
\subsection*{Moving changes between GIT remotes}
You can move changes between GITs in two ways; either you \emph{pull}
a branch from a remote or you \emph{push} a branch to a remote.

\subsubsection*{Pushing changes}
You can push changes to a remote through \\
\begin{center}\emph{git push <name> <branch>}\end{center}
If you omit the branch name, all branches will be pushed.
If changes you have done conflict with changes done in the remote
GIT, your push will be denied. You then have to \emph{pull} from the remote.
A push will also be denied if you push to a remote with a local working tree,
and you try to push to the branch currently checked out on the remote.

\subsubsection*{Pulling changes}
You can pull changes from a remote through \\
\begin{center}\emph{git pull $--$rebase <name> <branch>}\end{center}
If you omit the branch name, all branches will be pulled.
Please do not forget the rebase, or you can get yourself in trouble.
The pull model is more involved to explain,
and deemed outside the scope of this document.

If there are conflicts you have to resolve these.

\subsubsection*{Resolving conflicts}
If there are conflicts that GIT cannot resolve, you to step up.
A conflict is when GIT cannot resolve how to combine changes from
your local GIT and the remote GIT you are pulling from.
To see which files are in conflict you can use \emph{git status}.
Where there were conflicts you will have something that
looks like \lstinputlisting{conflict}
The first line is a starting marker. This is followed by lines that are
in conflict, these are from the local repository. Then a ===== marker,
followed by the lines in conflict from the remote repository. Finally
an ending marker and the revision from the remote. Change it into
what you want it to be and save the file. Remember that there may
be multiple blocks in conflict!

Add your changes through \emph{git add}, and continue the rebase
through \\
\begin{center}\emph{git rebase $--$continue}\end{center}
\newpage
%\subsection*{Kongull}
%When working on a remote machine, the two machines need to be able to
%find each other. This is complicated for laptops on roaming wireless
%connections as they do not have a static host name. Kongull, however,
%does have a static host name. For this reason it is easiest to only have
%Kongull as a remote on your laptop, and not the other way around.

%To avoid the restrictions on which branches you can push (see earlier
%discussion), we want to stick a bare clone in-between. This role
%can alternatively be served by a GIT hosted on a free service such
%as github.com.

%Here follows commands to setup this configuration on Kongull. You
%have to replace the placeholders.
%\lstinputlisting{kongull-setup-git}
%To move commits from your laptop to Kongull, you first push
%to the 'kongull' remote on your laptop, and then pull from 'origin' on Kongull
%in the \emph{myrepo} folder. To move changes the other way around, you
%first push to 'origin' on Kongull, and then pull from 'kongull' on your laptop.

\end{document}
